\documentclass[oneside,a4paper]{scrartcl}
\tolerance=9000

\author{Johannes Schneider, cedarsoft~GmbH~i.\,G}
\usepackage[pdftex,
bookmarks,bookmarksopen=true,bookmarksnumbered=true,
pdfauthor={Johannes Schneider},
pdfsubject={Swing MVC},
pdftitle={Swing MVC}]{hyperref}

\pagestyle{headings}

\usepackage{xcolor} 
\usepackage{listings}
\lstset{language=Java}

\usepackage[ngerman]{babel}
\usepackage[T1]{fontenc}
\usepackage[utf8]{inputenc}
\usepackage[right]{eurosym}
\usepackage{ae}
\usepackage[babel,german=quotes]{csquotes}

\title{Swing MVC}
\date{\today}



%Colors for listings.
\definecolor{listingBgColor}{rgb}{0.95,0.95,0.95}
\definecolor{colKeys}{rgb}{0,0,1}
\definecolor{colIdentifier}{rgb}{0,0,0}
\definecolor{colComments}{rgb}{0.5,0.5,0.5}
\definecolor{colString}{rgb}{0,0.5,0}

\lstset{%
    float=hbp,%
    basicstyle=\ttfamily\small, %
    identifierstyle=\color{colIdentifier}, %
    keywordstyle=\color{colKeys}, %
    stringstyle=\color{colString}, %
    commentstyle=\color{colComments}, %
    columns=flexible, %
    tabsize=2, %
    frame=single, %
    extendedchars=true, %
    showspaces=false, %
    showstringspaces=false, %
    numbers=left, %
    numberstyle=\tiny, %
    breaklines=true, %
    backgroundcolor=\color{listingBgColor}, %
    breakautoindent=true, %
    captionpos=b%
}


\begin{document}
\maketitle
\tableofcontents


\section{MVC allgemein}

\paragraph{}
Bereits seit etlicher Zeit hat sich das MVC-Prinzip als hilfreiches und oft eingesetztes Konstrukt
in der objektorientierten Welt durchgesetzt. 

Es ist dabei nicht immer einfach, die einzelnene
Komponenten zu identifizieren; wenn dies jedoch gelingt und eine solide Implementierung erfolgt, 
hat das MVC-Prinzip unschätzbare Vorteile, die sich insbesondere in einer sehr guten Wartbarkeit manifestieren. 
Der Hauptvorteil liegt aber wohl in einer nahezu optimalen Skalierbarkeit der Lösung.

Java hat bei seinen Swing-Komponenten bereits vor Jahren eine strikte Trennung von UI-Komponente und Model eingeführt.
Dies hat sich auch als richtig und gut erwiesen - auch wenn die Lernkurve am Anfang relativ steil ist.

\paragraph{}
Trotz der guten Erfahrungen mit MVC, finden sich interessanterweise kaum Projekte, die die gesamte UI und nicht
nur die einzelnen Komponenten nach diesem Prinzip gestalten.

Dies ist wahrscheinlich insbesondere der Tatsache geschuldet, dass es dafür von Sun kein fertiges Framework gibt.
Die Identifikation der einzelnen Komponenten ist aber nunmal eine nichttriviale Angelegenheit, zusätzlich
erfordert die Umsetzung initial einen erhöhten Aufwand. Dies führt dazu, dass heute oftmals die UIs fest
verdrahtet und damit kaum skalierbar gestaltet sind.

\paragraph{}
Ein Ansatz MVC auch im größeren Stile einzusetzen, soll hier dargestellt werden.
Dieser Ansatz wurde in einem Swing-Projekt mit rund 800 Klassen und einer sehr umfangreichen sowie flexiblen
GUI entwickelt und erfolgreich eingesetzt. Bei diesem Projekt hat sich insbesondere die Skalierbarkeit als
äußerst wertvoll erwiesen.



\section{Das Prinzip}
Entscheidend ist die Identifikation und Trennung der View vom Model. Der Controller ergibt sich in der Regel dann
von selbst. Initial kann er auch etwas vernachlässigt werden, da er relativ einfach refaktorisiert werden kann.

\subsection{Die View}
Bei einer komplexen Swing-Oberfläche besteht die View jeweils aus einer Sammlung von Swing-Komponenten. 
Bezeichnend für eine solches Sammlung ist, dass es in Java kein vorbereitetes Model dafür gibt. Für kleinste und kleine Projekte
ist dies auch nicht notwendig, bei einer etwas umfangreicheren Oberfläche dagegen sehr wohl.

Eine solche Sammlung könnte zum Beispiel die Menüleiste darstellen. Diese wird durch ein JMenuBar repräsentiert, welche
wiederum etliche JMenus enthält, die ihrerseits wieder andere JMenus oder JMenuItems (die einzelnen Einträge)
enthalten.

\subsection{Das Model}
\paragraph{}
Es bietet sich nun an, die Informationen über die Menüstruktur in einem eigenen Model zu halten und die gesamte
Menüleiste automatisch daraus zu generieren.

Das Model enthält dann die notwendigen Informationen um die einzelnen Menüs bzw. ihre Inhalte erstellen zu können.
Dies sind für die Menü-Einträge in der Regel Actions, für die Menüs die Informationen über Bezeichnung und Shortcut.

\paragraph{}
Dieses Model ist nun nicht mehr eindimensional (wie z.B. bei einer JList) sondern ist baumartig aufgebaut.
Die genaue Struktur des Baumes ist dabei nicht vorgegeben, da JMenus wiederum JMenus enthalten können.

\paragraph{}
Das Model kann dabei über völlig verschiedene Wege und zu frei wählbaren Zeitpunkten erstellt werden. So könnte die
Menüstruktur beispielsweise per XML vorgegeben, per Spring konfiguriert oder auch per Java-Code aufgebaut werden.

Bei einem korrekt implementierten Kontroller ist es auch problemlos möglich, zur Laufzeit weitere Menüs und Einträge
hinzuzufügen oder zu entfernen.

\subsection{Der Controller}
\paragraph{}
Der Controller ist in diesem Fall dafür verantwortlich die View aus dem Model zu erstellen und vor allem aktuell
zu halten. D.h. anhand der Struktur des Models muss eine entsprechende Struktur aus Swing-Komponenten erstellt und
diese mit dem Model verknüpft werden.

Bei Änderungen des Models (zum Beispiel das Hinzufügen oder Entfernen einzelner Menüs oder Einträge) muss die View
entsprechend angepasst werden.

Zusätzlich ist natürlich eine Verknüpfung der Swing-Komponenten mit den Actions aus dem Model notwendig.

\paragraph{}
Ist der Aufwand zur Erstellung des Controllers einmal gemacht, skaliert die Lösung sehr gut. Um neue, andere Menüs zu erstellen,
ist nur noch der Aufbau eines Models notwendig (zum Beispiel für andere Fenster oder als kontextsensitives Popup-Menü).




\section{Konkret}

\subsection{Generisches Model}
Es hat sich gezeigt, dass (nahezu) alle Swing-Oberflächen in einer baumartigen Struktur wiedergegeben werden können.
Deshalb bietet es sich an, ein generisches Modell zu erstellen, welches eben diese baumartige Struktur
optimal repräsentiert und die notwendige Funktionalität zur Benachrichtigung des Controllers im Falle von
strukturellen und inhaltlichen Änderungen bereit stellt.

\subsection{Generischer Controller}
Die Erstellung von Swing-Komponenten erfolgt jeweils nach einem ähnlichen Prinzip. Deshalb hat es sich als sehr
hilfreich erwiesen ein Presenter-Framework zu erstellen, welche die Swing-Komponenten anhand des Models erstellen
und vor allem aktuell halten.

Durch die große Gemeinsamkeit der Swing-Komponenten ist es möglich, den Großteil der Komplexität in abstrakten
Klassen zu halten und somit nur einmalig implementieren zu müssen.

\subsubsection{Presenter}
Für jeden Komponenten-Typ gibt es individuelle Presenter (z.B. für die JMenuBar), die genutzt werden. Im Model kann
dann bei Bedarf ein eigener Presenter hinterlegt werden, der dann benutzt wird. Sollte im Model kein eigener
Presenter (z.B. für ein JMenu) vorhanden sein, bringt jeder Presenter für seine Kinder einen Default-Presenter
mit (der ein Standard-Menü erstellt).

Es hat sich gezeigt, dass knapp 97\% der Fälle mit einem Standard-Presenter abgedeckt werden können. Die restlichen
Fälle werden durch die Möglichkeit der Bereitstellung eines individuellen Presenters im Model ohne Probleme
abgedeckt.




\section{Herausforderungen}
Die Herausforderungen bei der Umsetzung liegen insbesondere bei der erhöhten Komplexität des Models und vor allem
der Presenter.

Sind die Grundstrukturen für das Model und die Presenter einmal erstellt, können Erweiterungen jedoch sehr effizient
umgesetzt werden. Diese Grundstrukturen können auch über 

\subsection{Presenter}
Die Presenter müssen in der Lage sein, die entsprechenden Komponenten erstellen zu können. Zusätzlich ist es notwendig
die Kinder in der richtigen Reihenfolge hinzuzufügen. Sollten jetzt Kinder entfernt werden, muss die entsprechende
Komponente ebenfalls wieder aus der Hierarchie entfernt werden.

Dazu ist ein Mapping von Model zu Komponente notwendig und ebenso ein relativ umfangreiches System von Events, die vom
Model unterstützt werden müssen.

Zusätzlich muss bei Änderungen am Model die aktuelle View aktualisiert werden (wenn sich z.B. Labels oder die
Action verändern).

\subsection{Model}
Das Model muss eine baumartige Struktur repräsentieren und für jeden Knoten mehrere Objekte (wie z.B. Actions oder
Presenter) halten können. Außerdem muss gewährleistet sein, dass sämtliche Änderungen durch Events korrekt weiter
gemeldet werden, damit die Presenter die Darstellung korrekt vornehmen können.



\section{Umsetzung für Swing}
\paragraph{}
Dieses Prinzip ist natürlich grundsätzlich für viele Bereiche denkbar (z.B. auch SWT). Momentan besteht eine
Implementierung für Swing.

Diese Implmentierung besteht aus einem umfangreichen Model, welches die notwendige Funktionalität beinhaltet,
um eine baumartige Struktur mit entsprechendem Events darstellen zu können. In diesem Model können generisch
Objekte zu jeden Knoten abgelegt werden, so dass eine Erweiterung des Models nicht notwendig ist.

\paragraph{}
Es besteht außerdem eine Reihe von Presentern, die die wichtigsten Swing-Komponenten erstellen könnnen. Dabei
sind etliche abstrakte Basisklassen vorhanden, so dass die Implementierung eines neuen Presenters in wenigen
Zeilen Code - bei Action-basierten Komponenten sogar mit nur einer Zeile - abgeschlossen sind.


\section{Beispiel}
In diesem Beispiel wird ein einfaches JMenu mit vier Einträgen erstellt.

\subsection{Aufbau des Models}
Aufbau eines einfaches Models mit einem Root-Knoten und vier Kindern mit jeweils einer Action. Hier erfolgt der
Aufbau des Models per Java-Code. Es ist natürlich problemlos möglich, dies z.B. per Spring zu erledigen.

\begin{lstlisting}[language=Java,breaklines=true]
//Erstellt den Root-Knoten mit einer Action (die das Label vorgibt)
Node root = new DefaultNode( name, Lookups.singletonLookup( Action.class, menuAction ) );

//Fuegt jetzt vier Kinder hinzu
root.addChild( new DefaultNode( "1", Lookups.singletonLookup( Action.class, action0 ) ) );
root.addChild( new DefaultNode( "2", Lookups.singletonLookup( Action.class, action1 ) ) );
root.addChild( new DefaultNode( "3", Lookups.singletonLookup( Action.class, action2 ) ) );
root.addChild( new DefaultNode( "4", Lookups.singletonLookup( Action.class, action3 ) ) );
\end{lstlisting}

\subsection{Erstellung des Menüs}

\begin{lstlisting}[breaklines=true]
//Erstellt einen neuen Presenter
JMenuPresenter presenter = new JMenuPresenter();
//Mit Hilfe des Presenters kann jetzt das komplette Menue erstellt werden
JMenu menu = presenter.createPresentation( root );
\end{lstlisting}

Das Menü, welches mit Hilfe des Presenters erstellt wurde, ist nun mit dem Model verknüpft. Sollte also eine
Action verändert oder ausgetauscht werden, ein Eintrag hinzugefügt oder entfernt werden, wird das JMenu
sofort und ohne weitere Eingriffe aktualisiert.


\subsection{Erweiterungen}
Um jetzt einen eigenen Presenter einbauen zu können, wird dieser einfach in der Knotenstruktur mit abgelegt. Das
Model wird dann z.B. so aufgebaut:

\begin{lstlisting}[breaklines=true]
//Erstellt den Root-Knoten mit einer Action (die das Label vorgibt)
Node root = new DefaultNode( name, Lookups.singletonLookup( Action.class, menuAction ) );

//Fuegt jetzt vier Kinder hinzu
root.addChild( new DefaultNode( "1", Lookups.dynamicLookup( action0, myCustomPresenter ) ) );
root.addChild( new DefaultNode( "2", Lookups.singletonLookup( Action.class, action1 ) ) );
root.addChild( new DefaultNode( "3", Lookups.singletonLookup( Action.class, action2 ) ) );
root.addChild( new DefaultNode( "4", Lookups.singletonLookup( Action.class, action3 ) ) );
\end{lstlisting}

Die Implementierung von \enquote{myCustomPresenter} kann nun völlig frei gewählt werden und z.B. auch ein Untermenü mit
dynamisch generierten Einträgen erstellen.











\end{document}





















